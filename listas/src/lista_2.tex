\documentclass[12pt]{article}
\usepackage[portuguese]{babel}
\usepackage[utf8x]{inputenc}
\usepackage{amsmath,amssymb,amsthm,amsfonts}
\usepackage[T1]{fontenc}
\usepackage{fourier}
\usepackage[listings,skins,breakable]{tcolorbox}
\usepackage{ctable}

\begin{document}

\ \vspace{-1.4cm}
\begin{tcolorbox}[colback=black!0]
    \noindent
    \begin{minipage}{0.14\linewidth}
        \hspace*{-0.2cm}\includegraphics[height = 2.5cm]{UFSC.png}
    \end{minipage}
	\noindent
	\begin{minipage}{0.79\linewidth}
	    \begin{center}
	        \vspace*{0.2cm}
	        {\bf \large Universidade Federal de Santa Catarina} \\ \vspace{0.3cm}
			{\bf \large Centro de Ciências Físicas e Matemáticas} \\ \vspace{0.3cm}
			{\bf \large Departamento de Matemática}
		\end{center}
	\end{minipage}
\end{tcolorbox}

\noindent\textbf{Disciplina:} MTM5812 - H-Álgebra II\\
\textbf{Professora:} Melissa Weber Mendonça

\begin{center}
    \Large{2\textordfeminine\ Lista de Exercícios}
\end{center}

\section{Espaços Vetoriais}

\begin{enumerate}

\item Seja $V={\mathbb{R}}^2$. Então defina as seguintes operações:
  \begin{eqnarray*}
    (x,y) + (x_1,y_1) &=& (x+x_1,y+y_1)\\
    c(x,y) &=& (cx,y) \qquad (c\in \Re)
  \end{eqnarray*}
  Verifique se $V$ com estas operações é um espaço vetorial.
%
\item Em ${\mathbb{R}}^n$, defina duas operações
  \begin{eqnarray*}
    \alpha + \beta &=& \alpha - \beta\\
    c\alpha &=& -c\alpha
  \end{eqnarray*}
  onde as operações à direita são as operações usuais em $\mathbb{R}$. Quais axiomas dos espaços vetoriais são satisfeitos para este conjunto com estas operações?
%
\item Seja $V={\mathbb{R}}^2$ e considere
  \begin{eqnarray*}
    (x,y) + (x_1,y_1) &=& (x+x_1,0)\\
    c(x,y) &=& (cx,0) \qquad (c\in \Re)
  \end{eqnarray*}
  Verifique se $V$ com estas operações é um espaço vetorial.
% 
\item Quais dos seguintes conjuntos de vetores $\alpha =(\alpha_1,\ldots,\alpha_n) \in {\mathbb{R}}^n$ são subespaços de ${\mathbb{R}}^n$ ($n\geq 3$)?
  \begin{itemize}
  \item[(a)] $\{\alpha \in {\mathbb{R}}^n : \alpha_1\geq 0\}$ 
  \item[(b)] $\{\alpha \in {\mathbb{R}}^n : \alpha_1+3\alpha_2=\alpha_3\}$ 
  \item[(c)] $\{\alpha \in {\mathbb{R}}^n : \alpha_2=\alpha_1^2\}$ 
  \item[(d)] $\{\alpha \in {\mathbb{R}}^n : \alpha_1\alpha_2= 0\}$ 
  \end{itemize}
  % 
\item Seja $V$ o espaço de todas as funções $f:{\mathbb{R}} \rightarrow \mathbb{R}$. Quais dos seguintes subconjuntos são subespaços de $V$?
  \begin{itemize}
  \item[(a)] $\{f \in V : f(x^2) = f(x)^2\}$ 
  \item[(b)] $\{f \in V : f(0) = f(1)\}$ 
  \item[(c)] $\{f \in V : f(3) = 1+f(-5)\}$ 
  \item[(d)] $\{f \in V : f(-1)=0\}$ 
  \end{itemize}
  % 
\item Seja $V = {\mathbb{R}}^{n\times n}$, $n\geq 2$. Quais dos seguintes subconjuntos são subespaços de $V$?
  \begin{itemize}
  \item[(a)] $\{A \in V : A \mbox{ é inversível}\}$ 
  \item[(b)] $\{A \in V : A \mbox{ não é inversível}\}$ 
  \item[(c)] $\{A \in V : AB=BA, \mbox{ onde } B \mbox{ é uma matriz fixa em } V\}$ 
  \item[(d)] $\{A \in V : A^2=A\}$ 
  \end{itemize}
  % 
\item Seja $V$ o espaço de todas as funções $f:\mathbb{R} \rightarrow \mathbb{R}$. Seja $V_p$ o subconjunto de todas as funções pares, e $V_i$ o subconjunto de todas as funções ímpares.
  \begin{itemize}
  \item[(a)] Mostre que $V_p$ e $V_i$ são subespaços de $V$.
  \item[(b)] Mostre que $V_p+V_i=V$
  \item[(c)] Mostre que $V_p\cap V_i = \{0\}$.
  \end{itemize}
  % 
\item Verifique se os vetores
  \begin{align*}
    \alpha_1 &= (1,1,2,4) \\
    \alpha_2 &= (2,-1,-5,2)\\
    \alpha_3 &= (1,-1,-4,0)\\
    \alpha_4 &= (2,1,1,6)
  \end{align*}
  são linearmente independentes em ${\mathbb{R}}^4$. Em seguida, encontre uma base para o subespaço gerado por estes vetores.
  % 
\item Seja $V \subseteq \Re$ um espaço vetorial e suponha que $\alpha, \beta, \gamma \in V$ são l.i. Prove que $(\alpha + \beta)$, $(\beta +\gamma)$, $(\gamma +\alpha)$ são l.i.
  % 
\item Mostre que os vetores 
  \begin{align*}
    \alpha_1 &= (1,1,0,0)\\
    \alpha_2 &= (0,0,1,1)\\
    \alpha_3 &= (1,0,0,4)\\
    \alpha_4 &= (0,0,0,2)
  \end{align*}
  formam uma base para ${\mathbb{R}}^4$. Encontre as coordenadas de cada um dos vetores da base canônica nesta nova base ordenada $\{ \alpha_1,\alpha_2,\alpha_3,\alpha_4\}$.
  % 
\item Seja $W$ o subespaço de ${\cal{M}}(3,2)$ gerado por
  \begin{equation*}
    \begin{bmatrix}
      0& 0\\
      1 & 1\\
      0 & 0
    \end{bmatrix},
    \begin{bmatrix}
      0 & 1\\
      0 & -1\\
      1 & 0
    \end{bmatrix}
    \text{ e }
    \begin{bmatrix}
      0 & 1\\
      0 & 0\\
      0 & 0
    \end{bmatrix}.     
  \end{equation*}
  O vetor $
  \begin{bmatrix}
    0 & 2\\
    3 & 4\\
    5 & 0
  \end{bmatrix}$ pertence a $W$?
  
\item Seja $x = (x_1,x_2), y = (y_1,y_2) \in {\mathbb{R}}^2$ tais que
  \begin{align*}
    x_1y_1 + x_2y_2 &=& 0\\
    x_1^2+x_2^2=y_1^2+y_2^2&=&1.
  \end{align*}
  Mostre que ${\cal{B}} = \{x,y\}$ é uma base para ${\mathbb{R}}^2$. Encontre as coordenadas de um vetor $(a,b)$ nesta nova base ordenada. O que querem dizer geometricamente estas condições impostas a $x$ e $y$?
%
\item Seja ${\cal{P}}_2$ o conjunto de todos os polinômios reais a coeficientes reais de grau menor ou igual a 2. Seja $t\in \mathbb{R}$ um número real fixo e defina 
  \begin{align*}
    q_1(x) &= 1\\
    q_2(x) &= x+t\\
    q_3(x) &= (x+t)^2
  \end{align*}
  Prove que ${\cal{B}} = \{q_1,q_2,q_3\}$ é uma base para ${\cal{P}}_2$. Se
  $$f(x) = c_0 + c_1x+c_2x^2$$
  quais são as coordenadas de $f$ na base ${\cal{B}}$?
  
\item Considere o sistema linear
     \begin{equation*}
       \begin{cases}
         2x_1+4x_2-6x_3 & = a\\
         x_1-x_2+4x_3 &= b\\
         6x_2-14x_3 &= c
       \end{cases}
     \end{equation*}
     Seja 
     \begin{equation*}
       W = \{(x_1,x_2,x_3)\in {\mathbb{R}}^3 : (x_1,x_2,x_3) \text{ é solução do sistema.}\}.
     \end{equation*}
     Isto é, $W$ é o conjunto solução do sistema.
     \begin{itemize}
     \item[a)] Que condições devemos impor a $a$, $b$ e $c$ para que $W$ seja subespaço vetorial de ${\mathbb{R}}^3$?
     \item[b)] Nas condições determinadas em a), encontre uma base para $W$.
     \item[c)] Que relação existe entre a dimensão de $W$ e o grau de liberdade do sistema? Seria este resultado válido para quaisquer sistemas homogêneos?
     \end{itemize}

     \item Seja $U$ o subespaço de ${\mathbb{R}}^3$ gerado por $(1,0,0)$, e $W$ o subespaço de ${\mathbb{R}}^3$ gerado por $(1,1,0)$ e $(0,1,1)$. Mostre que ${\mathbb{R}}^3 = U \oplus W$.

     \item Sejam 
       \begin{align*}
         W_1 &= \{ (x,y,z,t) \in {\mathbb{R}}^4 : x+y=0 \text{ e } z-t=0\}\\
         W_2 &= \{(x,y,z,t) \in {\mathbb{R}}^4 : x-y-z+t=0\}
       \end{align*}
       subespaços de ${\mathbb{R}}^4$.
       \begin{itemize}
       \item[a)] Determine $W_1\cap W_2$.
       \item[b)] Exiba uma base para $W_1\cap W_2$.
       \item[c)] Determine $W_1+W_2$.
       \item[d)] $W_1+W_2$ é soma direta? Justifique.
       \item[e)] $W_1+W_2={\mathbb{R}}^4$?
       \end{itemize}

     \item Sejam
       \begin{align*}
         \beta &= \{(1,0),(0,1)\}\\
         \gamma &= \{(-1,1),(1,1)\}\\
         \rho &= \{(\sqrt{3},1),(\sqrt{3},-1)\}\\
         \nu &= \{(2,0),(0,2)\}
       \end{align*}
       bases ordenadas de ${\mathbb{R}}^2$.
       \begin{itemize}
       \item[a)] Ache as matrizes de mudança de base $[I]_\beta^\gamma$, $[I]_\gamma^\beta$, $[I]_\rho^\beta$, $[I]_\nu^\beta$.
       \item[b)] Quais são as coordenadas do vetor $v=(3,-2)$ em relação às quatro bases?
       \item[c)] As coordenadas de um vetor $v$ em relação à base $\gamma$ são dadas por
         \begin{equation*}
           [v]_\gamma = \begin{bmatrix} 4\\0\end{bmatrix}.
         \end{equation*}
         Quais são as coordenadas de $v$ em relação às outras bases?
       \end{itemize}

       \item Sejam $\beta_1 = \{(1,0),(0,2)\}$, $\beta_2 = \{(-1,0),(1,1)\}$ e $\beta_3 = \{(-1,-1),(0,-1)\}$ três bases ordenadas de ${\mathbb{R}}^2$.
         \begin{itemize}
           \item[a)] Ache $[I]_{\beta_1}^{\beta_2}$, $[I]_{\beta_2}^{\beta_3}$, $[I]_{\beta_1}^{\beta_3}$ e $[I]_{\beta_1}^{\beta_2} \cdot [I]_{\beta_2}^{\beta_3}$.
           \item[b)] Se for possível, ache uma relação entre estas matrizes de mudança de base.
           \end{itemize}
         \end{enumerate}
\end{document}

