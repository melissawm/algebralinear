\documentclass[12pt]{article}
\usepackage[portuguese]{babel}
\usepackage[utf8x]{inputenc}
\usepackage{amsmath,amssymb,amsthm,amsfonts}
\usepackage[T1]{fontenc}
\usepackage{fourier}
\usepackage[listings,skins,breakable]{tcolorbox}
\usepackage{ctable}

\begin{document}

\ \vspace{-1.4cm}
\begin{tcolorbox}[colback=black!0]
    \noindent
    \begin{minipage}{0.14\linewidth}
        \hspace*{-0.2cm}\includegraphics[height = 2.5cm]{UFSC.png}
    \end{minipage}
	\noindent
	\begin{minipage}{0.79\linewidth}
	    \begin{center}
	        \vspace*{0.2cm}
	        {\bf \large Universidade Federal de Santa Catarina} \\ \vspace{0.3cm}
			{\bf \large Centro de Ciências Físicas e Matemáticas} \\ \vspace{0.3cm}
			{\bf \large Departamento de Matemática}
		\end{center}
	\end{minipage}
\end{tcolorbox}

\noindent\textbf{Disciplina:} MTM5812 - H-Álgebra II\\
\textbf{Professora:} Melissa Weber Mendonça

\begin{center}
    \Large{3\textordfeminine\ Lista de Exercícios}
\end{center}

\begin{enumerate}
\item Prove que se $A,B : E\rightarrow F$ são transformações lineares e $\alpha \in {\mathbb{R}}$, então $A+B$ e $\alpha A$ são transformações lineares.

\item Seja $A:{\mathbb{R}}^2 \rightarrow {\mathbb{R}}^2$ a projeção sobre o eixo $x$, paralelamente à reta $y=ax$ ($a\ne 0$). Isto significa que, para todo $v=(x,y)$, temos que $Av = (x',0)$, tal que $Av-v$ pertence à reta $y=ax$. Exprima $x'$ em função de $x$ e $y$ e escreva a matriz de $A$ relativamente à base canônica do ${\mathbb{R}}^2$.

\item Dados os vetores $u_1 = (2,-1)$, $u_2 = (1,1)$ $u_3 = (-1, -4)$, $v_1 = (1,3)$, $v_2 = (2,3)$ e $v_3 = (-5,-6)$, decida se existe ou não um operador linear $A:{\mathbb{R}}^2 \rightarrow {\mathbb{R}}^2$ tal que $Au_1 = v_1$, $Au_2 = v_2$ e $Au_3 = v_3$. Mesma pergunta com $v_3 = (5,-6)$ e $v_3 = (5,6)$.

\item Seja $A:{\mathbb{R}}^2 \rightarrow {\mathbb{R}}^3$ uma transformação linear tal que $A(-1,1) = (1,2,3)$ e $A(2,3) = (1,1,1)$. Encontre a matriz de $A$ relativamente às bases canônicas do ${\mathbb{R}}^2$ e ${\mathbb{R}}^3$.

\item Quais das transformações abaixo são lineares?
  \begin{itemize}
  \item[a)] $A:{\mathbb{R}}^3 \rightarrow {\mathbb{R}}^3$, $(x,y,z) \mapsto (x,2^y, 2^z)$
  \item[b)] $A:{\mathbb{R}}^4 \rightarrow {\mathbb{R}}^3$, $(x,y,z,w) \mapsto (x-w, y-w, x+z)$
  \item[c)] $A:{\mathbb{R}}^{2\times 2} \rightarrow {\mathbb{R}}^{2\times 2}$, $\begin{bmatrix} a & b\\c & d\end{bmatrix} \mapsto \begin{bmatrix} a & c\\b & d\end{bmatrix}$
  \item[d)] $A:{\cal{M}}_{2\times 2} \to {\mathbb{R}}$, $\begin{bmatrix} a & b\\c & d\end{bmatrix} \mapsto \text{det}\begin{bmatrix} a & b \\c & d\end{bmatrix}$
  \item[e)] $A:{\mathbb{R}} \to {\mathbb{R}}$, $x\mapsto |x|$.
  \end{itemize}

\item Se $R(x,y) = (2x,x-y,y)$ e $S(x,y,z) = (y-z,z-x)$, ache $R\circ S$ e $S\circ R$.

\item Seja $A:{\mathbb{R}}^3 \rightarrow {\mathbb{R}}^3$ dado por $A(x,y,z) = (ay+bz,cz,0)$. Mostre que $A^3=0$.

\item Dado o operador $A:{\mathbb{R}}^2\rightarrow {\mathbb{R}}^2$ com $A(x,y)=(3x-2y,2x+7y)$, ache um vetor não nulo $v=(x,y)$ tal que $Av=5v$.

\item Sejam $A,B:E\rightarrow E$ operadores lineares. Suponha que existam vetores $u,v\in E$ tais que $Au$ e $Av$ sejam linearmente dependentes. Prove que $BAu$ e $BAv$ também são linearmente dependentes.

\item Seja $A:E\rightarrow E$ um operador linear. Para quaisquer vetores $u\in {\cal{N}}(A)$ e $v\in $Im$(A)$, prove que se tem $Au \in {\cal{N}}(A)$ e $Av \in $Im$(A)$.

\item Escreva a expressão de um operador $A:{\mathbb{R}}^2 \rightarrow {\mathbb{R}}^2$ cujo núcleo seja a reta $y=x$ e cuja imagem seja a reta $y=2x$.

\item Defina um operador $A:{\mathbb{R}}^2 \rightarrow {\mathbb{R}}^2$ que tenha como núcleo e imagem o eixo $x$.

\item Considere a transformação linear $A:{\mathbb{R}}^4\rightarrow {\mathbb{R}}^3$ dada por
  $$A(x,y,z,t) = (x+y+z+2t,x-y+2z,4x+2y+5z+6t).$$
  Encontre um vetor $b\in {\mathbb{R}}^3$ que não pertença à imagem de $A$ e com isso exiba um sistema linear de três equações com quatro incógnitas que não tem solução.

\item Determine uma base para a imagem e uma base para o núcleo de cada uma das transformações lineares abaixo e indique quais são sobrejetivas:
  \begin{itemize}
  \item[(a)] $A:{\mathbb{R}}^2 \rightarrow {\mathbb{R}}^2$, $A(x,y) = (x-y,x-y)$
  \item[(b)] $B:{\mathbb{R}}^4\rightarrow {\mathbb{R}}^4$, $B(x,y,z,t) = (x+y,z+t,x+z,y+t)$
  \item[(c)] $C:{\mathbb{R}}^3 \rightarrow {\mathbb{R}}^3$, $C(x,y,z) = (x+\frac{y}{2},y+\frac{z}{2},z+\frac{x}{2})$
  \item[(d)] $E:{\cal{P}}_n \rightarrow {\cal{P}}_{n+1}$, $E(p(x)) = xp(x)$.
  \end{itemize}

\item Dê, quando possível, exemplos de transformações lineares satisfazendo:
\begin{itemize}
\item[a)] $T:{\mathbb{R}}^3\to {\mathbb{R}}^2$ sobrejetora
\item[b)] $T:{\mathbb{R}}^3\to {\mathbb{R}}^2$ com ${\cal{N}}(T) = \{0\}$
\item[c)] $T:{\mathbb{R}}^3\to {\mathbb{R}}^2$ com Im$(T) = \{0\}$
\item[d)] $T:{\mathbb{R}}^2\to {\mathbb{R}}^2$ com ${\cal{N}}(T) = \{ (x,y)\in {\mathbb{R}}^2 \, | \, x=y\}$
\item[e)] $T:{\mathbb{R}}^3\to {\mathbb{R}}^3$ com ${\cal{N}}(T) = \{ (x,y,z) \in {\mathbb{R}}^3 \, | \, z=-x\}$
\end{itemize}
   
\item Sem fazer hipóteses sobre as dimensões de $E$ e $F$, sejam $A:E\rightarrow F$ e $B:F\rightarrow E$ transformações lineares. Se $AB$ é inversível, prove que $A$ é sobrejetiva e $B$ é injetiva.

\item Prove ou dê um contra-exemplo: Se $A,B:E\rightarrow F$ são operadores de mesmo posto $r$, então o produto $BA$ tem posto $r$.

\item Seja $A:E\rightarrow E$ fixo, e seja $C(A)$ o conjunto de todos os operadores lineares $X:E\rightarrow E$ que comutam com $A$ (ou seja, $AX=XA$). Prove que $C(A)$ é um subespaço vetorial e que se $X,Y\in C(A)$, então $XY \in C(A)$.

\item Seja $E=C^0({\mathbb{R}})$ o espaço das funções contínuas $f:{\mathbb{R}}\rightarrow {\mathbb{R}}$. Defina o operador linear $A:E\rightarrow E$ que associa, a cada $f\in E$, $Af=\varphi$, onde 
  $$\varphi (x) = \int_0^x f(t) dt, \quad x\in {\mathbb{R}}.$$
  Determine o núcleo e a imagem do operador $A$.

\item Prove que os operadores lineares $E_{11}, E_{12}, E_{21}, E_{22} : {\mathbb{R}}^2 \rightarrow {\mathbb{R}}^2$, definidos por $E_{11}(x,y) = (x,0)$, $E_{12}(x,y) = (0,x)$, $E_{21}(x,y) = (y,0)$ e $E_{22}(x,y) = (0,y)$ constituem uma base para o espaço vetorial ${\cal{L}}({\mathbb{R}}^2)$. Prove ainda que outra base deste espaço pode ser formada com os operadores $A, B, C, I$, onde $A(x,y) = (x+3y,y)$, $B(x,y)=(x,0)$, $C(x,y)=(x+y,x-y)$ e $I(x,y)=(x,y)$.

\item Seja $v$ um vetor não nulo de um espaço vetorial $E$, de dimensão finita. Dado qualquer espaço vetorial $F\ne \{0\}$, mostre que existe uma transformação linear $A:E\rightarrow F$ tal que $Av\ne 0$.
  
\item Seja $A:E\rightarrow E$ um operador nilpotente (isto é, existe $k_0 \in \mathbb{N}$ tal que $A^k = 0$ para todo $k\geq k_0$). Prove que existe algum vetor $v\ne 0$ em $E$ tal que $Av=0$. 

\item Seja $0_{k \times \ell }$ a matriz nula em ${\mathbb{R}}^{k\times \ell}$. Sejam $A\in {\mathbb{R}}^{m\times m}$ uma matriz de posto $r$ e $B\in {\mathbb{R}}^{n\times n}$ uma matriz de posto $s$. Prove que a matriz 
  $$\begin{pmatrix}
    A & 0_{m\times n}\\
    0_{n\times m} & B
  \end{pmatrix} \in {\mathbb{R}}^{(m+n) \times (m+n)}$$
  tem posto $r+s$.
\end{enumerate}
\end{document}
